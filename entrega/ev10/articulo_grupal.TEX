\documentclass[12pt]{article}
\usepackage[utf8]{inputenc}
\usepackage[spanish, english]{babel}
\usepackage{apacite}
\usepackage{setspace}
\usepackage{geometry}
\geometry{margin=1in}

% Document settings
\setlength{\parindent}{0.5in}
\doublespacing

\begin{document}

% Title Section
\title{Desarrollo de Software de Gestión para Fincas Recolectoras de Café (RECOFFEE)}
\author{Marlon Estiven Torres Medina\\
Facultad SENA\\
marlontorresmedina@gmail.com\\
\\
Juan David Garcia Calderon\\
jgarciacalderon396@gmail.com}
\date{}
\maketitle

\selectlanguage{english}

% Abstract Section
\begin{abstract}
This study focuses on the development and analysis of management software designed specifically for coffee farms, with the aim of digitizing the operational and administrative processes that are part of the daily operations in these farms. The software not only simplifies but also optimizes crucial activities such as harvesting, allowing precise control of work through the creation of profiles, daily assignment of tasks, and detailed recording of production in terms of kilos of coffee.

The system not only facilitates the management of workers but also generates real-time data essential for a more rigorous and transparent control of the company's overall productivity. It also introduces key functionality that allows managers to accurately and centrally record daily production in kilos collected. This directly feeds into an integrated financial management system, where costs associated with production are calculated automatically, eliminating the need for manual calculations.

The software also provides real-time financial visualization, allowing managers to monitor cash flow and other key financial indicators, improving decision-making and responsiveness to market fluctuations or changes in the company's operation. This is particularly relevant for small and medium-sized producers who have often been limited by the lack of access to advanced technological tools to manage their resources.

In conclusion, the software not only seeks to improve the operational efficiency of coffee farms but also seeks to positively impact their long-term profitability. By reducing the costs associated with manual management and optimizing the use of resources, these companies can increase production without significantly increasing their investments. Digitalization is considered a key tool to modernize the sector and promote greater competitiveness and sustainability, especially among small and medium-sized producers seeking to adapt to the increasingly demanding demands of the market.
\end{abstract}

% Keywords Section
\textbf{Palabras claves:}
\begin{itemize}
    \item Digitalizaci\'on agr\'icola
    \item Gesti\'on de fincas
    \item Recolecci\'on de caf\'e
    \item Optimizaci\'on de recursos
    \item Innovaci\'on tecnol\'ogica
    \item Software agr\'icola
    \item Productividad cafetalera
    \item Automatizaci\'on de procesos
    \item Sostenibilidad en el caf\'e
    \item Rentabilidad agr\'icola
    \item Exportaci\'on de caf\'e
    \item Peque\~nos productores
\end{itemize}

\selectlanguage{spanish}

% Introduction Section
\section{Introducci\'on}
El sector agr\'icola, y en particular el de las fincas recolectoras de caf\'e, ha experimentado hist\'oricamente una dependencia de procesos manuales en su gesti\'on diaria. La tecnolog\'ia, sin embargo, ha comenzado a ganar terreno en esta \'area, abriendo posibilidades para mejorar la eficiencia, optimizar los recursos y promover un crecimiento m\'as sostenible. En este contexto, surge la necesidad de un software especializado que aborde las particularidades de la recolecci\'on de caf\'e y facilite una administraci\'on m\'as precisa y eficaz.

El desarrollo de un software de gesti\'on para fincas recolectoras de caf\'e es una soluci\'on que busca no solo digitalizar el proceso de recolecci\'on, sino tambi\'en proporcionar una herramienta integral que permita una mejor administraci\'on de los recursos financieros y productivos. Esto es especialmente importante para peque\~nos y medianos productores, quienes suelen enfrentar desaf\'ios al depender de m\'etodos tradicionales de registro y contabilidad.

A lo largo de este art\'iculo, se analizar\'a el desarrollo del software RECOFFEE, desde su conceptualizaci\'on hasta su implementaci\'on, y se destacar\'an los beneficios operativos que ofrece. Adem\'as, se examina el impacto potencial en la rentabilidad de las fincas al integrar esta herramienta en su d\'ia a d\'ia.

\section{Desarrollo del Software: Caracter\'isticas y Funcionalidades}
\subsection{Gesti\'on de recolectores y actividades diarias}

El software RECOFFEE es un componente fundamental para gestionar eficientemente los recolectores de caf\'e y las actividades diarias que realizan en la finca. Este m\'odulo est\'a dise\~nado para facilitar la administraci\'on de uno de los recursos m\'as importantes en cualquier finca cafetalera: la mano de obra. Cada recolector cuenta con un perfil personalizado, creado y gestionado \'integramente por el administrador de la finca. Este perfil incluye datos personales relevantes, tales como el nombre, los d\'ias trabajados, las tareas asignadas y el historial de recolecci\'on, lo que permite un control m\'as detallado y estructurado de las operaciones diarias.

El sistema otorga al administrador de la finca la responsabilidad de ingresar esta informaci\'on con precisi\'on, aprovechando el acceso a la informaci\'on inmediatamente y minimizar errores comunes en la toma manual de datos. El seguimiento continuo de la producci\'on diaria permite tener un control exhaustivo sobre el rendimiento de cada recolector. A pesar de los reportes generados autom\'aticamente por el sistema, el administrador puede obtener una visi\'on clara y detallada de cu\'antos kilos de caf\'e ha recolectado cada trabajador en un d\'ia, una semana o a lo largo de la temporada de cosecha.

El registro hist\'orico de las actividades diarias permite analizar el desempe\~no individual de los recolectores y identificar patrones de productividad general. Esta informaci\'on es invaluable para planificar futuras cosechas, ajustes en la asignaci\'on de tareas y mejorar la distribuci\'on del trabajo en funci\'on de las capacidades de cada recolector y las condiciones de la finca.

\subsection{Registro de kilos}

Otra funcionalidad clave de RECOFFEE es el registro de los kilos de caf\'e recolectados. A trav\'es de una interfaz intuitiva, los administradores pueden ingresar la cantidad de caf\'e que cada recolector ha cosechado diariamente. Esta informaci\'on se almacena de manera segura y se puede consultar en cualquier momento, lo que facilita la generaci\'on de informes de producci\'on y la identificaci\'on de tendencias.

El registro autom\'atico de los kilos recolectados no solo permite un control m\'as preciso de la producci\'on diaria, sino que tambi\'en alimenta el sistema financiero de la finca. Esto elimina la necesidad de c\'alculos manuales y reduce la posibilidad de errores, lo que ahorra tiempo y mejora la precisi\'on en la administraci\'on de la finca.

\subsection{Visualizaci\'on y seguimiento financiero en tiempo real}

Uno de los mayores beneficios del software es la capacidad de visualizar y seguir el estado financiero de la finca en tiempo real. A medida que se registran los kilos recolectados, el sistema calcula autom\'aticamente los costos asociados a la producci\'on, lo que permite a los administradores obtener una visi\'on clara y actualizada de las finanzas.

Esta funci\'on es accesible desde cualquier dispositivo con conexi\'on a Internet, lo que brinda a los administradores la flexibilidad de monitorear el estado financiero de la finca desde cualquier lugar y en cualquier momento. Este acceso a informaci\'on financiera en tiempo real facilita la toma de decisiones r\'apidas y acertadas, mejorando la transparencia y la eficiencia en la gesti\'on de los recursos.

\subsection{Optimizaci\'on de Recursos y Eficiencia Operativa}

La digitalizaci\'on de procesos agr\'icolas no solo se trata de modernizaci\'on, sino de la mejora directa en la eficiencia operativa. A trav\'es de RECOFFEE, las fincas pueden optimizar tanto el tiempo como los recursos. Al dejar atr\'as los m\'etodos manuales de registro, el software reduce considerablemente el riesgo de errores y la p\'erdida de datos.

Uno de los aspectos clave es la reducci\'on del tiempo que los administradores dedican a tareas administrativas, permiti\'endoles centrarse en decisiones estrat\'egicas. La capacidad de automatizar procesos cr\'iticos, como el c\'alculo de pagos o la generaci\'on de informes, libera tiempo que se puede invertir en optimizar la operaci\'on diaria de la finca. As\'i mismo, al centralizar todos los datos en un \'unico sistema, se mejora la accesibilidad y la organizaci\'on de la informaci\'on, lo que resulta en una mayor eficiencia operativa.

\subsection{Mejora en la toma de decisiones}

El acceso en tiempo real a datos tanto operativos como financieros proporciona a los administradores una ventaja significativa en la toma de decisiones. En lugar de esperar a que se generen informes manualmente o de recopilar informaci\'on dispersa, los administradores pueden consultar el estado actual de la finca en cualquier momento. Esto les permite reaccionar r\'apidamente a cualquier variaci\'on en la producci\'on o en las finanzas, tomando decisiones m\'as r\'apidas y acertadas.

Adem\'as, la capacidad de analizar los datos almacenados en el software permite una toma de decisiones m\'as informada. Los administradores pueden identificar patrones y tendencias en la producci\'on y los costos, lo que facilita la planificaci\'on a largo plazo y la optimizaci\'on de los recursos disponibles.

\subsection{Reducción de Costos y Aumento de Rentabilidad}

La implementación de RECOFFEE no solo tiene como objetivo mejorar la eficiencia operativa, sino que también contribuye a reducir los costos generales de operación. Al optimizar los procesos de recolección y gestión financiera, las fincas pueden reducir el desperdicio de recursos y mejorar la utilización de sus activos.

El monitoreo en tiempo real de los flujos de caja y la producción permite a los administradores detectar y corregir ineficiencias de manera oportuna, lo que minimiza las pérdidas. Como resultado, se reducen los costos generales de operación, lo que a su vez contribuye a un aumento en la rentabilidad de la finca.

\section{Cómo nace la idea}

“La idea de desarrollar RECOFFEE surgió de la necesidad crítica que enfrentan las fincas recolectoras de café en cuanto al manejo y registro de la producción diaria. Tradicionalmente, los finqueros han dependido de métodos manuales, utilizando hojas de cuadernos sueltas para llevar el registro de los kilos recolectados. Este método, aunque familiar, es altamente susceptible a errores y pérdidas de información, lo que puede resultar en una gestión ineficiente y costosa. La solución propuesta con RECOFFEE es transformar este proceso arcaico en un sistema digital confiable y eficiente, ofreciendo una especie de cuaderno virtual que almacena y organiza toda la información de manera segura y accesible.

\subsection{Motivación Personal y Profesional}

El nacimiento de RECOFFEE no fue solo un esfuerzo técnico, sino también una respuesta a vivencias personales. Habiendo vivido en el campo durante más de 14 años, en una finca recolectora de café, pude observar de primera mano los desafíos que enfrentan las pequeñas fincas. Me di cuenta de que, mientras que las grandes fincas logran vender su café a buen precio, a menudo gracias a su volumen y reconocimiento en el mercado, muchas pequeñas fincas producen café de alta calidad pero no consiguen los mismos beneficios debido a su tamaño y falta de visibilidad. Esta situación crea una gran desigualdad en el mercado, donde las pequeñas fincas, a pesar de su arduo trabajo y excelente producto, luchan por sobrevivir.

\subsection{Desarrollo Tecnológico}

El desarrollo de RECOFFEE pasó por varias etapas cruciales, comenzando con la planificación y selección de tecnologías adecuadas. Optamos por utilizar C# para el backend debido a su robustez y fiabilidad, Angular para el frontend por su capacidad para crear interfaces de usuario dinámicas y responsivas, y SQL Server para la gestión de bases de datos, garantizando una administración eficiente y segura de la información.

La idea central de RECOFFEE no es solo limitarse a la gestión de procesos, sino también convertirse en una plataforma integral donde las fincas puedan vender sus productos y colaborar directamente con exportadoras de café. De esta manera, las pequeñas fincas podrán acceder a mercados más amplios y obtener mejores precios por su producto, promoviendo así una economía más justa y sostenible.

\subsection{Impacto y Proyecciones Futuras}

La visión a largo plazo para RECOFFEE es ambiciosa. Imaginamos un futuro en el que la tecnología de gestión de fincas se convierta en una herramienta esencial para todas las fincas recolectoras de café, independientemente de su tamaño. Con la implementación de RECOFFEE, esperamos no solo resolver problemas inmediatos de gestión y registro, sino también empoderar a los pequeños productores, brindándoles las herramientas necesarias para competir en igualdad de condiciones con los grandes productores.

Además, al proporcionar una solución tecnológica avanzada, buscamos fomentar la adopción de prácticas agrícolas más sostenibles y eficientes. La digitalización y la automatización no solo mejorarán la rentabilidad de las fincas, sino que también contribuirán a la conservación del medio ambiente al optimizar el uso de recursos y minimizar el desperdicio.” – Marlon Torres (desarrollador principal)

\section{Historia y Contexto de la Recolección de Café}

La recolección de café no es simplemente una tarea agrícola; es una tradición rica en historia y cultura. Desde los tiempos de nuestros abuelos, la cosecha de café ha sido una actividad esencial. Recordemos que Colombia es un país cafetero, el café hace parte de nuestra cultura, tiene un lugar en los corazones de nuestras familias y generaciones. Aunque la tecnología avanza a pasos agigantados, muchos recolectores todavía confían en métodos tradicionales, que si bien son entrañables, pueden ser ineficientes y propensos a errores.

\subsection{Los desafíos del pasado}

Tenemos a Juan, un finquero lidiando con una lluvia de hojas de cuaderno sueltas mientras trata de anotar cuántos kilos de café se han recolectado ese día. Los registros se mojan, se pierden o, peor aún, se los come un perro “como le sucede a la mayoría de niños con su tarea”. Todo esto puede llevar a grandes pérdidas económicas y una administración menos transparente.

\subsection{La solución tecnológica}

Aquí es donde entra en nuestro software, listo para salvar el día. Con este software, los finqueros pueden decir adiós a las hojas sueltas y hola a un sistema organizado y seguro que mantiene toda la información en un solo lugar. Es como pasar de escribir cartas a mano a enviar correos electrónicos instantáneamente, pero mucho más sabroso, porque estamos hablando de café y, seamos sinceros, ¿a quién no le gusta una buena taza de café? Despertarse y ser lo primero que se toma, amigo mío, eso es vida: disfrutar de su aroma, de su sabor, de su color. Les pido una disculpa, pero adoro el café y estar trabajando en algo que puede ayudar a esas fincas me llena de orgullo.

\subsection{Implementación y Beneficios}

\subsubsection{Facilidad de Uso}

La implementación de RECOFFEE no requiere un título en ingeniería espacial. Está diseñado para ser intuitivo y fácil de usar, incluso para aquellos que no se llevan bien con la tecnología. Al digitalizar el proceso de recolección, los finqueros pueden acceder a la información en tiempo real y desde cualquier lugar, lo que simplifica enormemente la gestión diaria.

\subsubsection{Impacto en la Comunidad}

Además de mejorar la eficiencia y la precisión, RECOFFEE tiene el potencial de transformar comunidades enteras. Al empoderar a los pequeños productores con herramientas modernas, se promueve un crecimiento económico más equitativo y sostenible. Los productores pueden obtener mejores precios por su café, lo que a su vez mejora la calidad de vida de sus familias y comunidades.

\subsubsection{Proyecciones Futuras}

\paragraph{Expansión de Funcionalidades}

RECOFFEE no se detiene en la recolección. En futuras versiones, planeamos incluir funcionalidades para ayudar a los finqueros a vender su café directamente a exportadores, eliminando intermediarios y mejorando sus márgenes de ganancia. También estamos considerando integrar herramientas de análisis predictivo que puedan anticipar problemas de producción y sugerir soluciones antes de que ocurran.

\paragraph{Sostenibilidad}

Un aspecto crucial de RECOFFEE es su contribución a la sostenibilidad. Al optimizar los recursos y minimizar el desperdicio, ayudamos a las fincas a operar de manera más ecológica. Esto es esencial en un mundo donde el cambio climático y la conservación del medio ambiente son temas cada vez más importantes.

\section{Conclusiones}

La digitalización es un paso crucial para modernizar las fincas recolectoras de café y mejorar su eficiencia. El desarrollo de software como RECOFFEE ofrece una solución integral para gestionar tanto la producción como las finanzas, optimizando recursos y mejorando la rentabilidad. Aunque no se incluyeron funcionalidades relacionadas con tiendas en el desarrollo actual del software, este sigue siendo una herramienta robusta que contribuye significativamente al éxito de las fincas.

La capacidad de gestionar recolectores, registrar kilos recolectados y monitorear las finanzas en tiempo real proporciona a los administradores las herramientas necesarias para tomar decisiones más informadas y eficientes. En resumen, la implementación de RECOFFEE representa un avance significativo hacia la modernización del sector agrícola y abre nuevas oportunidades para el crecimiento sostenible y rentable de las fincas recolectoras de café.



\end{document}
